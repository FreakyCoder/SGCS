\section{Introduction}
Complex systems, most often biological systems, often exhibit what is known as emergent behavior. Emergent behavior refers to an observable behavior of a system constrained only by rules of the environment(environmental conditions) and, more importantly, the rules that each participant follows independently. Such systems are capable of collectively accomplishing tasks that no individual would be able to do alone. Moreover, some of these are able to function without expensive communication to a central command center. As in \parencite{marsh_demystification_2009}, expensive could refer to time or another resource like energy. Ant colonies are one of the most well-known examples of a system that exhibits emergent behavior, where participants in the system can exchange information outside of the central hub(nest). They are capable of task allocation, deciding between nest maintenance, foraging, and patrolling, using only environmental and social cues with no central authority, shown in \parencite{gordon_organization_1996}. Furthermore, according to \parencite{adler_information_1992}, the way an ant colony performs such tasks is also demonstrative of emergent behavior. Once again there is no apparent coordination between individual ants, yet they are collectively able to achieve their goal.
\par Stigmergy is a form of indirect communication through the modification of the environment, through which emergent behavior can be achieved. In essence, modifications of the environment made by an individual can be detected by other participants to obtain information. As per \parencite{theraulaz_brief_1999}, French zoologist Pierre-Paul Grass\'e introduces the concept in 1959 to explain how the observed coordination of insects' activities emerges from independent actions of the individual.
\par In \parencite{theraulaz_brief_1999}, two main types of stigmergy have been identified. Quantitative stigmergy refers to the use of a single type of stimuli and its quantity affects the response probability of the individual. Qualitative stigmergy, on the other hand, refers to stimuli that vary in type and can thus provoke a set of actions by the individual, depending on both the type and quantity of sensed stimuli.
\par As stated in \parencite{carde_encyclopedia_2009} \todo{better source}, trail pheromones, a form of stigmergy, are used by a variety of species in nature such as some ants and bees. When an individual discovers a resource, they lay a trail of pheromones while returning to the nest or hive. Other individual can then follow those trails to reach the resource. This is especially practical when the retrieval of said resource cannot be performed by the individual alone due to its size, for example. Furthermore, information can be encoded in the pheromone trail. Resource quantity or proximity can be indicated by an intensification of the trail. Hence, a pheromone map is created around the colony's nest, which is used by patrollers and foragers for navigation.
\par The concept of emergent behavior is already finding uses in technology. Swarm robotics is a subfield of robotics that draws inspiration from seemingly simplistic non-intelligent creatures that are able to achieve wonders through collaboration as explained in \parencite{schranz_swarm_2020}. Here we focus particularly on its application for area surveillance, assessing its tolerance to individual failures and its scalability. 
