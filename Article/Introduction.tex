\section{Introduction}
Complex systems, most often biological ones, often exhibit what is known as emergent behavior. Emergent behavior refers to an observable behavior of a system constrained only by rules of the environment(environmental conditions) and, more importantly, the rules that each participant follows independently. Such systems are capable of collectively accomplishing tasks that no individual would be able to do alone. Moreover, some of these are able to function without expensive communication to a central command center. As in \parencite{marsh_demystification_2009}, expensive could refer to time or another resource like energy. 
\par The concept of emergent behavior is already finding uses in technology. As per \parencite{schranz_swarm_2020}, swarm robotics is a subfield of robotics that draws inspiration from seemingly simplistic non-intelligent creatures that are able to achieve wonders through collaboration.
\par Ant colonies are one of the most well-known examples of a system that exhibits emergent behavior, where individuals can exchange information outside of the central hub(nest). They are capable of task allocation, deciding between nest maintenance, foraging, and patrolling, using only environmental and social cues with no central authority, shown in \parencite{gordon_organization_1996}. Furthermore, according to \parencite{adler_information_1992}, the way an ant colony performs such tasks is also demonstrative of emergent behavior. Once again they are collaboratively able to achieve their goal.
\par Stigmergy is a form of indirect communication through the modification of the environment, through which emergent behavior can be achieved. In essence, modifications of the environment made by an individual can be detected by other participants to obtain information. As per \parencite{theraulaz_brief_1999}, French zoologist Pierre-Paul Grass\'e introduces the concept in 1959 to explain how the observed coordination of insects' activities emerges from independent actions of the individual.
\par In \parencite{theraulaz_brief_1999}, two main types of stigmergy have been identified. Quantitative stigmergy refers to the use of a single type of stimuli and its quantity affects the response probability of the individual. Qualitative stigmergy, on the other hand, refers to stimuli that vary in type and can thus provoke a set of actions by the individual, depending on both the type and quantity of sensed stimuli.
\par As stated in \parencite{robinson_decay_2008}, trail pheromones, a form of stigmergy, are used by a variety of ant species in nature. When an individual discovers a resource, they lay a trail of pheromones. Other individuals can then follow those trails to reach the resource. Furthermore, information can be encoded in the pheromone trail. Resource quantity or proximity can be indicated by an intensification of the trail. Hence, a pheromone map is created around the colony's nest, which is used by patrollers and foragers for navigation. 
\par Here we focus on the robustness, or how well the system responds to failures of agents, and scalability, i.e. how does an increase of the number of agents affect performance, of an ant-inspired approach to swarm robotics coordination for the task of area exploration, using our Swarm Guidance and Communication System(SGCS). 
