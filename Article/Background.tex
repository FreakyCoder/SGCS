\section{Background}
\par Swarm organization systems that do not rely on emergent behavior do exist. For instance, \textit{Karma}, presented in \parencite{dantu_programming_2011}, utilizes a beehive model, where participants with very simple software and hardware are governed by a central computer. Only when an individual is attached to the central hub does communication occur between the two. Hence, expensive and energy-demanding communication hardware is avoided. Although \parencite{dantu_programming_2011} demonstrates some adaptibility, the lack of in-field communication is a limiting factor. Furthermore, a centralized hub is a single point of failure, reducing robustness even if the system is tolerant to individual failures. 
\par In \parencite{elston_hierarchical_2008}, a heterogenous swarm of mother- and daughterships performs search and tracking. The motherships perform task allocation among each other through negotiation, and then each one coordinates its subswarm of daughterships.
\par \parencite{marino_fault-tolerant_2009} presents a patrolling system and showcases how it adapts to individual failures.
In \parencite{flint_cooperative_2002}, a dynamic programming solution algorithm for swarm coordination is used.
\par As per \parencite{schranz_swarm_2020}, true emergent behavior remains rare in practice. Industrial applications still by and large rely on centralized control even if basic swarm behaviors are integrated, and thus the system is referred to as a ``swarm".
\par According to \parencite{calvo_bio-inspired_2011}, nonbioinspired coordination of swarms robust to individual failures do exist but do not cover the environment completely, whereas purely mathematical strategies are unable to cope with agent failure.
\par Random walk methods are evaluated and improved upon in \parencite{pang_swarm_2019}.
\par Stigmergic emergent behavior can be used in robotics for building strucutes like in \parencite{werfel_designing_2014}. The individuals use the current state of the structure they are building to guide their behavior. However, this concept is obviously inapplicable to area surveillance.
\par Instead, variations of the concept of pheromones are most common in this domain even though exceptions like \parencite{flint_cooperative_2002} do exist. \parencite{salman_phormica_2020} and \parencite{hutchison_antbots_2010} present similar approaches to trail pheromones. Each robot has the ability to project UV light on photochromic material, and thus leave artificial pheromones, which can then be detected by other individuals of the swarm. \parencite{fujisawa_designing_2014} uses ethanol instead. However, these environmental modifications rely on a controlled environment, limiting their pratical applications. Similarly, \parencite{arvin_cos_nodate} and \parencite{na_bio-inspired_2021} use cameras and LCD screens. In \parencite{song_novel_2020}, a novel neural network model for foraging is proposed but along with \parencite{calvo_bio-inspired_2011} assumes indirect environmental communication.
\par According to \parencite{hunt_testing_2019}, a more common approach is the use of digital or virtual pheromones, shared globally throughout the swarm. For example, \parencite{winkelstrater_virtual_2019} and \parencite{ravankar_bio-inspired_2016} demonstrate such systems by using a centralized synchronization node, which maintains a global pheromone map. Nevertheless, this approach makes impractical assumptions such as infinite communication range or relies on some sort of centralization, negating much of the benefits of a truly distributed swarm system. 
\par Various works such as \parencite{fossum_repellent_2014} and \parencite{schroeder_efficient_2017} rely on either of the former two approaches.
\par \parencite{payton_pheromone_2001}, \parencite{pearce_using_2006}, and \parencite{schmickl_trophallaxis_2006} avoid the generation of a map; instead, they only rely on peer-to-peer communication. In the former two, the swarm agents themselves act as pheromones of sorts. In \parencite{li_pheromone-inspired_2019}, pre-set communication nodes are used to establish communication within the swarm.
\par \parencite{hutchison_digital_2005} considers military appilcations and proposes the concept of \textit{place agents}, representing parts of the physical space and the strength of each flavor of pheromone in it, and thus the graph of these \textit{place agents} is a virtual map. \textit{Walker agents} represent the swarm individuals and can move from one \textit{place agent} to another. The map is assumed to be globally synchronized between \textit{walker agents}. Furthermore, \parencite{sauter_performance_2005} notes that a fixed pattern search covers an area faster than its pheromone-guided counterpart. 
\par In \parencite{hauert_ant-based_2008}, a method for deployment of an ad hoc wireless communication network of UAVs between 2 ground users is presented. Pheromones are deposited on swarm agents themselves due to the lack of positioning information, needed for a virtual map. 
\par \parencite{kuiper_mobility_2006} utilizes the concept of local virtual pheromone maps that are shared when agents are within communication range of one another. Likewise, \parencite{parunak_swarming_2003} makes use of this concept in combination with task allocation for target search and imaging. \parencite{pack_developing_2005} utilizes a similar approach even if pheromones are not explicitly mentioned. All of the above, however, are concerned with relatively small swarm sizes with 0\% individual failure chance. 
\par In \parencite{tinoco_pherocom_2022}, local virtual pheromone maps are also used for surveillance of an indoor area. The effect of the communication range is explored but for a maximum of 36 agents. Furthermore, 0\% individual failure chance is assumed. Similarly, \parencite{nguyen_improving_2021} uses a genetic algorithm to optimize swarm parameters for communication of incomplete virtual pheromone maps. \parencite{hecker_beyond_2015} also utilizes a genetic algorithm for foraging and, additionally, to account for sensor errors.
\par Adaptive fault recovery strategies for swarms are discussed in \parencite{oladiran_fault_nodate}. Collective fault detection is demonstrated in \parencite{christensen_fireflies_2009}. Byzantine fault tolerance is implemented in \parencite{liao_uav_2021}.
\par \parencite{bjerknes_fault_2013} examines the reliability of a swarm performing flocking and beacon-taxis, whereas \parencite{winfield_safety_2006} employs Failure Mode and Effect Analysis.
\par Finally, \parencite{hunt_testing_2019} suggests that repellent pheromone robotic swarm systems are not scalable, i.e. the efficiency decreases at high number of participants due to pheromone saturation and is even comparable to random walk algorithms.
