\section{Simulation}
\label{sec:sim}
A 2D simulation is developed to assess the performance of SGCS and specifically its robustness. The following parameters can be set at the start: simulation steps($T$), area width($W$) and height($H$), number of robots($n_r$), decison steps($N$), considered directions($n$), communication range($R_{comm}$), sensor range($R_{sens}$), individual malfunction chance of each robot($P(F)$), speed($v$), pheromone drop steps($N_p$), initial pheromone strength($S_{initial}$), fence strength coefficient($\alpha$), and pheromone decay rate($\lambda$). 
The simulation performs the following steps:
\begin{enumerate}
	\item Execute the main algorithm for each robot as described in Section \ref{sec:alg}.
	\item The distance between each pair of individuals $d_{ij}$ is calculated and compared with the communication range $R_{comm}$. If $d_{ij}<R_{comm}$, the pair synchronizes their virtual pheromone maps to simulate communication with limited range.
	\item For each robot, randomly decide whether to destroy it, based on $P(F)$. 
\end{enumerate}
Simulations were conducted for 10, 20, 50, 100, and 500 robots. For each, experiments were conducted with either 0\%, 0.1\%, 1\%, and 5\% failure chance on each timestep or with pre-set times for a certain number of agents to fail.
The same simulations were also performed for an unbiased random walk that simply picks a random future direction $\vec{e_f}$  with $\theta < 90^{\circ}$(the angular distance between $\vec{e_f}$ and the current direction $\vec{e_c}$) every $N$ steps.
