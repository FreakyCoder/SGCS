%! TEX TS-program = xelatex
\documentclass[a4paper, 11pt]{article}
\usepackage[backend=biber, style=ieee]{biblatex}
\usepackage[tmargin=0.49in, bmargin=1.81in, lmargin=1.56in, rmargin=1.56in]{geometry}
\usepackage{titlesec}
\usepackage{xcolor}
\usepackage{enumitem}
\usepackage{multicol}
\addbibresource{references.bib}
\setlist[enumerate]{label*=\arabic*.}
\newcommand{\todo}[1]{\textcolor{orange}{TODO: #1}}
\makeatletter
\renewcommand{\maketitle}{\bgroup\setlength{\parindent}{0pt}
\begin{flushleft}
  \Large\@title \par
  \large \textit{\@author}
\end{flushleft}\egroup
}
\makeatother

\makeatletter
\def\abstract{\noindent\it\textbf{\abstractname:}}
\makeatother

\makeatletter
\def\keywords{\noindent\it\textbf{Keywords:}}
\makeatother

\titleformat{\section}
  {\normalfont\large}{\thesection.}{1em}{}
\titleformat{\subsection}
  {\normalfont\large}{\thesubsection.}{1em}{}
\titleformat{\subsubsection}
  {\normalfont\large}{\thesubsubsection.}{1em}{}
\setlength{\parindent}{0pt}

\title{Robustness and scalability of incomplete virtual pheromone maps for stigmergic collective exploration}
\author{
Kaloyan Dimitrov$^{1}$, Vladimir Hristov$^{2}$\\
\begin{footnotesize}
$^{1}$American College of Sofia\\
$^{2}$Technical University of Sofia, dept. Automation of electric drive, faculty of Automatics\\
Emails: $^{1}$k.dimitrov25@acsbg.org $^{2}$vdhristov@tu-sofia.bg
\end{footnotesize}
}
\date{}

\begin{document}
\maketitle
\begin{abstract}
Swarm Guiding and Communication System(SGCS) is a decision-making and information-sharing framework for robot swarms that only needs close-range peer-to-peer communication and no centralized control. Each robot makes decisions based on an incomplete virtual pheromone map that is updated on each interaction with another robot, imitating ant colonial behavior. Similar systems rely on continuous communication with no range limitations, environment modification, or centralized control. A computer simulation is developed to assess the effectiveness and robustness of the framework in covering an area, according to the following parameters: number of robots, communication range, individual malfunction chance of each robot, speed, battery discharge rate, and pheromone decay rate. \todo{Include results}
\end{abstract}
\begin{keywords}
swarm, exploration, pheromones, robustness, scalability
\end{keywords}
\section{Introduction}
Complex systems, most often biological ones, often exhibit what is known as emergent behavior. Emergent behavior refers to an observable behavior of a system constrained only by rules of the environment(environmental conditions) and, more importantly, the rules that each participant follows independently. Such systems are capable of collectively accomplishing tasks that no individual would be able to do alone. Moreover, some of these are able to function without expensive communication to a central command center. As in \parencite{marsh_demystification_2009}, expensive could refer to time or another resource like energy. 
\par The concept of emergent behavior is already finding uses in technology. As per \parencite{schranz_swarm_2020}, swarm robotics is a subfield of robotics that draws inspiration from seemingly simplistic non-intelligent creatures that are able to achieve wonders through collaboration.
\par Ant colonies are one of the most well-known examples of a system that exhibits emergent behavior, where individuals can exchange information outside of the central hub(nest). They are capable of task allocation, deciding between nest maintenance, foraging, and patrolling, using only environmental and social cues with no central authority, shown in \parencite{gordon_organization_1996}. Furthermore, according to \parencite{adler_information_1992}, the way an ant colony performs such tasks is also demonstrative of emergent behavior. Once again they are collaboratively able to achieve their goal.
\par Stigmergy is a form of indirect communication through the modification of the environment, through which emergent behavior can be achieved. In essence, modifications of the environment made by an individual can be detected by other participants to obtain information. As per \parencite{theraulaz_brief_1999}, French zoologist Pierre-Paul Grass\'e introduces the concept in 1959 to explain how the observed coordination of insects' activities emerges from independent actions of the individual.
\par In \parencite{theraulaz_brief_1999}, two main types of stigmergy have been identified. Quantitative stigmergy refers to the use of a single type of stimuli and its quantity affects the response probability of the individual. Qualitative stigmergy, on the other hand, refers to stimuli that vary in type and can thus provoke a set of actions by the individual, depending on both the type and quantity of sensed stimuli.
\par As stated in \parencite{robinson_decay_2008}, trail pheromones, a form of stigmergy, are used by a variety of ant species in nature. When an individual discovers a resource, they lay a trail of pheromones. Other individuals can then follow those trails to reach the resource. Furthermore, information can be encoded in the pheromone trail. Resource quantity or proximity can be indicated by an intensification of the trail. Hence, a pheromone map is created around the colony's nest, which is used by patrollers and foragers for navigation. 
\par Here we focus on the robustness, or how well the system responds to failures of agents, and scalability, i.e. how does an increase of the number of agents affect performance, of an ant-inspired approach to swarm robotics coordination for the task of area exploration, using our Swarm Guidance and Communication System(SGCS). 

\section{Background}
\par Swarm organization systems that do not rely on emergent behavior do exist. For instance, \textit{Karma}, presented in \parencite{dantu_programming_2011}, utilizes a beehive model, where participants with very simple software and hardware are governed by a central computer. Only when an individual is attached to the central hub does communication occur between the two. Hence, expensive and energy-demanding communication hardware is avoided. Although \parencite{dantu_programming_2011} demonstrates some adaptibility, the lack of in-field communication is a limiting factor. Furthermore, a centralized hub is a single point of failure, reducing robustness even if the system is tolerant to individual failures. 
\par In \parencite{elston_hierarchical_2008}, a heterogenous swarm of mother- and daughterships performs search and tracking. The motherships perform task allocation among each other through negotiation, and then each one coordinates its subswarm of daughterships.
\par \parencite{marino_fault-tolerant_2009} presents a patrolling system and showcases how it adapts to individual failures.
In \parencite{flint_cooperative_2002}, a dynamic programming solution algorithm for swarm coordination is used.
\par As per \parencite{schranz_swarm_2020}, true emergent behavior remains rare in practice. Industrial applications still by and large rely on centralized control even if basic swarm behaviors are integrated, and thus the system is referred to as a ``swarm".
\par According to \parencite{calvo_bio-inspired_2011}, nonbioinspired coordination of swarms robust to individual failures do exist but do not cover the environment completely, whereas purely mathematical strategies are unable to cope with agent failure.
\par Random walk methods are evaluated and improved upon in \parencite{pang_swarm_2019}.
\par Stigmergic emergent behavior can be used in robotics for building strucutes like in \parencite{werfel_designing_2014}. The individuals use the current state of the structure they are building to guide their behavior. However, this concept is obviously inapplicable to area surveillance.
\par Instead, variations of the concept of pheromones are most common in this domain even though exceptions like \parencite{flint_cooperative_2002} do exist. \parencite{salman_phormica_2020} and \parencite{hutchison_antbots_2010} present similar approaches to trail pheromones. Each robot has the ability to project UV light on photochromic material, and thus leave artificial pheromones, which can then be detected by other individuals of the swarm. \parencite{fujisawa_designing_2014} uses ethanol instead. However, these environmental modifications rely on a controlled environment, limiting their pratical applications. Similarly, \parencite{arvin_cos_nodate} and \parencite{na_bio-inspired_2021} use cameras and LCD screens. In \parencite{song_novel_2020}, a novel neural network model for foraging is proposed but along with \parencite{calvo_bio-inspired_2011} assumes indirect environmental communication.
\par According to \parencite{hunt_testing_2019}, a more common approach is the use of digital or virtual pheromones, shared globally throughout the swarm. For example, \parencite{winkelstrater_virtual_2019} and \parencite{ravankar_bio-inspired_2016} demonstrate such systems by using a centralized synchronization node, which maintains a global pheromone map. Nevertheless, this approach makes impractical assumptions such as infinite communication range or relies on some sort of centralization, negating much of the benefits of a truly distributed swarm system. 
\par Various works such as \parencite{fossum_repellent_2014} and \parencite{schroeder_efficient_2017} rely on either of the former two approaches.
\par \parencite{payton_pheromone_2001}, \parencite{pearce_using_2006}, and \parencite{schmickl_trophallaxis_2006} avoid the generation of a map; instead, they only rely on peer-to-peer communication. In the former two, the swarm agents themselves act as pheromones of sorts. In \parencite{li_pheromone-inspired_2019}, pre-set communication nodes are used to establish communication within the swarm.
\par \parencite{hutchison_digital_2005} considers military appilcations and proposes the concept of \textit{place agents}, representing parts of the physical space and the strength of each flavor of pheromone in it, and thus the graph of these \textit{place agents} is a virtual map. \textit{Walker agents} represent the swarm individuals and can move from one \textit{place agent} to another. The map is assumed to be globally synchronized between \textit{walker agents}. Furthermore, \parencite{sauter_performance_2005} notes that a fixed pattern search covers an area faster than its pheromone-guided counterpart. 
\par In \parencite{hauert_ant-based_2008}, a method for deployment of an ad hoc wireless communication network of UAVs between 2 ground users is presented. Pheromones are deposited on swarm agents themselves due to the lack of positioning information, needed for a virtual map. 
\par \parencite{kuiper_mobility_2006} utilizes the concept of local virtual pheromone maps that are shared when agents are within communication range of one another. Likewise, \parencite{parunak_swarming_2003} makes use of this concept in combination with task allocation for target search and imaging. \parencite{pack_developing_2005} utilizes a similar approach even if pheromones are not explicitly mentioned. All of the above, however, are concerned with relatively small swarm sizes with 0\% individual failure chance. 
\par In \parencite{tinoco_pherocom_2022}, local virtual pheromone maps are also used for surveillance of an indoor area. The effect of the communication range is explored but for a maximum of 36 agents. Furthermore, 0\% individual failure chance is assumed. Similarly, \parencite{nguyen_improving_2021} uses a genetic algorithm to optimize swarm parameters for communication of incomplete virtual pheromone maps. \parencite{hecker_beyond_2015} also utilizes a genetic algorithm for foraging and, additionally, to account for sensor errors.
\par Adaptive fault recovery strategies for swarms are discussed in \parencite{oladiran_fault_nodate}. Collective fault detection is demonstrated in \parencite{christensen_fireflies_2009}. Byzantine fault tolerance is implemented in \parencite{liao_uav_2021}.
\par \parencite{bjerknes_fault_2013} examines the reliability of a swarm performing flocking and beacon-taxis, whereas \parencite{winfield_safety_2006} employs Failure Mode and Effect Analysis.
\par Finally, \parencite{hunt_testing_2019} suggests that repellent pheromone robotic swarm systems are not scalable, i.e. the efficiency decreases at high number of participants due to pheromone saturation and is even comparable to random walk algorithms.

\section{Applications}
\label{subsec:applications}
One of the main intended applications is in agriculture. As the bee population is dwindling, such MAV(Micro-aerial vehicle) swarms can prove a suitable replacement and help with sustainability. Moreover, closed-space hydroponics and aeroponics systems currently rely on manual pollination---a task that can be automated with MAV swarms. 
\par Some plant species need to be pollinated in bursts due to their short bloom period. Currently, this is achieved by moving bee hives to the desired location, but this could also be achieved with a robotic swarm. All this shows the substantial improvements to agriculture that such a system could bring.
\par Artificial swarms can also be incredibly useful in search-and-rescue scenarios. Having a fault-tolerant system that quickly covers a wide area, even in difficult conditions, could be the difference between life and death.   
\par A robotic swarm could be used to sweep a battlefield and discover mines and bombs, greatly reducing the risk for teams that dispose of unexploded ordnance.
\par Identifying radiation, chemical, and biological hazards is another dangerous task that could be effectively performed by artificial swarms. If employed, such a system could protect the health and lives of professionals in the field. 
\par Another possible application is security. A swarm of unpredictably moving tiny robots eliminates blind spots of stationary cameras. This can in turn greatly aid law enforcement and justice. 
\par Exploration of hard-to-reach areas is another task well-suited to artificial swarms.
\par Such a swarm could be used to track migrations of all kinds of animals, regardless of the conditions of their habitat. It could also be used to track and reduce water pollution.
\par Moreover, it could be used to find and track endangered species, helping with their preservation. Finally, lost farm animals can also be located with such a system. 
\par The simulation in particular could help determine optimal parameters for the system when used in practice.  


\section{Approach and Algorithm}
SGCS uses partial virtual pheromone maps that are only updated when two participants come within communication range of each other. No direct environment modification or central control is used. This approach has the following advantages: 
\begin{enumerate}
\item No preparation of the site of operation is needed prior to deploying the system.
\item No specialized hardware is needed for environment modification or detection of those modifications.
\item Energy consumption for communication is reduced due to the low range requirements.
\item The system is agnostic to the communication hardware, allowing adaptibility to the environment where the system is deployed. For example, RF communication could be used for aerial swarms and sonar for underwater operations.
\item No single point of failure exists as failure of any individual participant does not effect the others.
\end{enumerate}
The algorithm consists of repeatedly peforming the following steps:
	\begin{enumerate}
		\item For each pheromone, decrease its strength, according to its decay rate, and destroy it if it has completely decayed. 
		\item Every $N$ interations, decide on a direction by picking $n$ random directions and for the $k$-th($0\le k<n$) direction calculating a desirability rating $d_k$:
		\begin{enumerate}
			\item Calculate future position $\vec{r_k}=\vec{r} + \vec{\Delta r}$, where $\vec{r}$ is the current position and $\vec{\Delta r}$ is the possible future translation($|\vec{\Delta r}|=N v \Delta t$). $\Delta t$ is the average time between iterations and $v$ is the average speed of the robot. 
			\item If $\frac{|\vec{r_k}|}{v} r_{bdcg} < B\%-10\%$(B\% is the battery level), $d_k=0$ because the robot cannot return home with $10\%$ safety margin.
			\item Otherwise, with a virtual pheromone map of $m$ pheromones $$d_k = \sum_{i=1}^m {a_i|\vec{r_k}-\vec{r_{i}}|}$$ $a_i$ is the strength of the $i$-th pheromone and $\vec{r_{i}}$ is its position.
		\end{enumerate}
	\item Pick the direction with the highest $d_k$, or if $d_k=0$ $\forall k \in [1;n]$, pick the direction to return to the starting position.
	\end{enumerate}
A simulation with Python is developed to assess the performance of SGCS. The following parameters can be adjusted at the start: number of robots($n_r$), communication range($s_{comm}$), individual malfunction chance of each robot($P(F)$), speed($v$), battery discharge rate($r_{bdcg}$), and pheromone decay rate($r_{pdcy}$). 
The simulation performs the following steps:
\begin{enumerate}
	\item For each robot, randomly decide whether to destroy it, based on $P(F)$. 
	\item Simulate the draining of the battery of each robot.
	\item Destroy every robot whose battery is fully drained.
	\item Execute the main algorithm for each robot.
	\item The distance between each pair of individuals $s_{ij}$ is calculated and compared with the communication range $s_{comm}$. If $s_{ij}<s_{comm}$, the pair synchronizes their virtual pheromone maps to simulate communication with limited range.
\end{enumerate}
\par The algorithm, if implemented in practice, has a complexity of $O(n^2)$ with respect to the number of considered future positions and the number of pheromones on the virtual map of the robot. The simulation has a complexity of $O(n^3)$ as the above steps need to be performed for each robot.


\section{Simulation}
\label{sec:sim}
A 2D simulation is developed to assess the performance of SGCS and specifically its robustness. The following parameters can be set at the start: simulation steps($T$), area width($W$) and height($H$), number of robots($n_r$), decison steps($N$), considered directions($n$), communication range($R_{comm}$), sensor range($R_{sens}$), individual malfunction chance of each robot($P(F)$), speed($v$), pheromone drop steps($N_p$), initial pheromone strength($S_{initial}$), fence strength coefficient($\alpha$), and pheromone decay rate($\lambda$). 
The simulation performs the following steps:
\begin{enumerate}
	\item Execute the main algorithm for each robot as described in Section \ref{sec:alg}.
	\item The distance between each pair of individuals $d_{ij}$ is calculated and compared with the communication range $R_{comm}$. If $d_{ij}<R_{comm}$, the pair synchronizes their virtual pheromone maps to simulate communication with limited range.
	\item For each robot, randomly decide whether to destroy it, based on $P(F)$. 
\end{enumerate}
Simulations were conducted for 10, 20, 50, and 100 robots. For each, experiments were conducted with either 0\%, 0.1\%, and 1\% failure chance on each timestep or with pre-set times for a certain number of agents to fail. Simulation parameters are listed before each respective experiment.

\section{Results and Analysis}
\label{sec:res}

\section{Future Improvements} 
\subsection{Algorithm}
The algorithm currently uses a constant speed. Varying the speed of the robot depending on some factors could lead to an increase in efficiency while reducing power consumption. 
\par No constraints are imposed on the change of direction by the robots, i.e. the algorithm relies on the robots being able to instantaneously change direction up to $90^{\circ}$, which is impossible in practice. The algorithm should factor in the maneuverability of the robots.
\par Currently each robot only performs observations and data-collection. Working on a task like pollination(See Section \ref{subsec:applications}), would require a modification of the algorithm that allows the robot to stay in place for some duration to perform the task, for example.
\par Only repulsive pheromones are used as of now. However, attractive pheromones could be implemented to foster collaboration when performing tasks, similar to what is observed in nature as noted in \parencite{david_morgan_trail_2009}. Hence, although currently quantitative stigmergy is implemented, qualitative could be as well.
\par Data collection and distribution throughout the network is not considered here. Ant colonial behavior in nature could possibly be used to fulfill this task similar to \parencite{adler_information_1992}.
\par The algoirthm currently does not avoid collisions between two participants, which is vital for practical applications.
\subsection{Simulation}
The simulation is currently only 2D, and can thus only simulate ground swarms. However, as explained in Section \ref{subsec:applications}, SGCS could be particularly useful for aerial or underwater swarms. Hence, a 3D simulation would be needed to assess the feasibility of those use cases.
\par No obstacles are currently present in the simulation. The addition of obstacles would provide a more realistic environment for testing of the algorithm.
\par The simulation of communication is also oversimplified. It is instantaneous and fully reliable if the pair of robots are within range of each other or is completely absent otherwise. Using a more unreliable and thus realistic model for communication simulation would allow to better assess the algorithm's practical feasibility. Additionally, making obstacles affect the communication would be a further improvement.
\subsection{Hardware Implementation}
\label{subsec:hwreq}
The system has not yet been deployed on real hardware. For a swarm of robots to be able to make use of SGCS, each individual must meet the following requirements:
\begin{enumerate}
	\item Processing capabilities are needed for execution of the above outlined algorithm. Little processing power is needed due to the simplicity of the algorithm. 
	\item Memory capabilities are necessary to record the virtual pheromone map.
	\item Positioning is needed to place pheromones on the virtual pheromone map as well as to compare the robot's current position with the recorded pheromones.
	\item Peer-to-peer communication is required to allow for synchronization of virtual pheromone maps between participants. Short-range communication is sufficient (See Section \ref{sec:sim}). The effect of communication range on performance is not explored in this work and needs to be investigated further. 
\end{enumerate}
\par A number of research platforms for testing swarm robotics algorithms are reviewd in \parencite{schranz_swarm_2020}, most of which are suitable for SGCS due to the low requirements outlined above. Testing on a hardware platform would provide a more accurate measure of the algorithms efficiency, robustness, and scalability. Moreover, SGCS was initially inspired by the development of the RoboBee, and the need for swarm coordination on a heavily restricted hardware platform such as \parencite{chen_controlled_2019}. As such, it is our goal to one day have SGCS perform practical tasks on real hardware.


\section{Conclusion}
SGCS aims to orchestrate a robotic swarm to survey an area by emulating ant colonial behavior. The algorithm has little hardware requirements to accomplish this task due to the simplicity of the software. A simulation is developed to assess effectiveness and robustness. Results are presented and analyzed. However, the system is still untested on real hardware. A multitude of possible future improvements are presented. 
\printbibliography
\end{document}
