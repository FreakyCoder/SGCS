\documentclass[a4paper, 12pt, twocolumn, times]{article}
\usepackage[style=ieee, backend=biber]{biblatex}
\usepackage[a4paper, margin=1in]{geometry}
\usepackage{xcolor}
\addbibresource{references.bib}
\newcommand{\todo}[1]{\textcolor{orange}{TODO: #1}}

\author{Kaloyan Dimitrov}

\title{Swarm Guiding and Communication System(SGCS)}
\date{\today}

\begin{document}
\maketitle
\begin{abstract}
SGCS is a decision-making and information-sharing framework for robot swarms that only needs close-range peer-to-peer communication and no centralized control. Each robot makes decisions based on an incomplete virtual pheromone map that is updated on each interaction with another robot, imitating ant colonial behavior. Similar systems rely on continuous communication with no range limitations, environment modification, or centralized control. A computer simulation is developed to assess the effectiveness and robustness of the framework in covering an area, according to the following parameters: number of robots, communication range, individual malfunction chance of each robot, speed, battery discharge rate, and pheromone decay rate. \todo{Include results}
\end{abstract}
\section{Introduction}
Complex systems, most often biological systems, often exhibit what is known as emergent behavior. Emergent behavior refers to an observable behavior of a system constrained only by rules of the environment(environmental conditions) and, more importantly, the rules that each participant follows independently. Such systems are capable of collectively accomplishing tasks that no individual would be able to do alone. Moreover, some of these are able to function without expensive communication to a central station. Expensive could refer to time or another resource like energy \parencite{marsh_demystification_2009}. Ant colonies are one of the most well-known examples of a system that exhibits emergent behavior, where participants in the system can exchange information outside of the central hub(nest). They are capable of task allocation - deciding between nest maintenance, foraging, and patrolling using only environmental and social cues with no central authority \parencite{gordon_organization_1996}. Furthermore, the way an ant colony performs each of these tasks is also demonstrative of emergent behavior. Once again there is no apparent coordination between individual ants, yet they are collectively able to achieve their goal.
\par The concept of emergent behavior is already finding uses in technology. For instance, swarm robotics is a subfield of robotics that draws inspiration from seemingly simplistic non-intelligent creatures that are able to achieve wonders through collaboration \parencite{schranz_swarm_2020}.
\section{Background}
\par Swarm organization systems that do not rely on emergent behavior do exist. For instance, \textit{Karma} utilizes a beehive model, where participants with very simple software and hardware are governed by a central computer. Only when an individual is attached to the central "hive" does communication occur between the two. Hence, expensive and energy-demanding communication hardware is avoided. \todo{Include cost scaling for swarms.}This, however, limits the in-field adaptibility of the collective. Furthermore, a centralized "hive" is a single point of failure, reducing robustness even if the system is tolerant to individual failures.
\par Stigmergy is a form of indirect communication through the modification of the environment, through which emergent behavior can be achieved. In essence, modifications of the environment made by an individual can be detected by other participants to obtain information. French zoologist Pierre-Paul Grass\'e introduces the concept in 1959 to explain how the observed coordination of insects' activities emerges from independent actions of the individual \parencite{theraulaz_brief_1999}.
\par Two main types of stigmergy have been identified:
\begin{enumerate}
\item Quantitative
\item Qualitative
\end{enumerate}
\par Trail pheromones, a form of stigmergy, are used by a variety of species in nature such as some ants and bees. When an individual discovers a resource, they lay a trail of pheromones while returning to the nest or hive. Other individual can then follow those trails to reach the resource \parencite{carde_encyclopedia_2009}. This is especially practical when the retrieval of said resource cannot be performed by the individual alone due to its size, for example. Furthermore, information can be encoded in the pheromone trail. Resource quantity or proximity can be indicated by an intensification of the trail \parencite{carde_encyclopedia_2009}. Hence, a pheromone map is created around the colony's nest, which is used by patrollers and foragers for navigation.
\par An application of this concept in the field of robotics is \textit{Phormica}. Each robot has the ability to project UV light on photochromic material, and thus leave artificial pheromones, which can then be detected by other individuals of the swarm \parencite{salman_phormica_2020}. However, this environmental modification relies on a controlled environment, limiting its pratical applications.\todo{Add other examples of stigmergy in robotics}
\par An alternative approach is the use of virtual pheromone maps. A positioning system is used and the location of each pheromone is recorded in memory. Synchronization is performed either through peer-to-peer communication, in which case each participant maintains its own map, or a centralized map is used, and the participants only update it and repeatedly retrieve it for decision-making \parencite{winkelstrater_virtual_2019}. Uninterrupted communication over an unlimited range is required for both approaches to keep the pheromone map up to date.\todo{Add ant touch communication}
\section{Applications}
A swarm of robots has a variety of uses.
\par One of the main intended applications is in agriculture. As the bee population is dwindling, such MAV(Micro-aerial vehicle) swarms can prove a suitable replacement and help with sustainability. Moreover, closed-space hydroponics and aeroponics systems currently rely on manual pollination - a task that can be automated with MAV swarms. Furthermore, some plant species need to be pollinated in bursts due to their short bloom period. Currently, this is achieved by moving bee hives to the desired location, but this could also be achieved with a robotic swarm. All this shows the substantial improvements to agriculture that such a system could bring.
\par Artificial swarms can also be incredibly useful in search-and-rescue scenarios. Having a fault-tolerant system that quickly covers a wide area, even in difficult conditions, could be the difference between life and death. \todo{Reference results of repeatedly covered area.}  
\par Furthermore, a robotic swarm could be used to sweep a battlefield and discover unexploded mines and bombs. This greatly reduces the death-risk of teams that dispose of unexploded ordnance.
\par Identifying radiation, chemical, and biological hazards is another dangerous task that could be effectively performed by artificial swarms. If employed, such a system could protect the health and lives of professionals in the field. 
\par Another possible application is security. A swarm of unpredictably moving tiny robots provides better area coverage compared to stationary surveillance cameras that can be avoided by passing through blind spots. This can in turn greatly help law enforcement and justice. 
\par Exploration of hard-to-reach areas is another task well-suited to artificial swarms.
\par Furthermore, such a swarm could be used to track migrations of all kinds of animals, living under water and/or on land. It could also be used to track and reduce water pollution.
\par Moreover, it could be used to find and track endangered species, helping with their preservation. Finally, lost farm animals can also be located with such a system. 
\par The simulation in particular could help determine optimal parameters for the system when used in practice.  
\section{Conclusion}
SGCS aims to orchestrate a robotic swarm to survey an area by emulating ant colonial behavior. The simple model I have come up with has little hardware and software requirements to accomplish this task. While SGCS provides individual robots with the ability to make informed decisions based on their environment, the system is still unusable. Factors such as expected movement time(based on the remaining energy in the power source) need to be considered in order to improve the decision-making process and not lose operating robots that are low on energy. In order to prevent robots from being pushed away from their starting positions, pheromone decay needs to be added and, more importantly, carefully chosen. The system also currently does not avoid robot collisions as it only takes the decision-making is only based on the distance from the origin point. Further, the system currently has no way to report information it discovers to the central hub. There are multiple suitable bioalgorithms that can be used to fulfill this task \parencite{adler_information_1992}.
\printbibliography
\end{document}

