\section{Approach and Algorithm}
\label{sec:alg}
SGCS uses partial virtual pheromone maps that are only updated when two participants come within communication range of each other. No direct environment modification or central control is used. This approach has the following advantages: 
\begin{enumerate}
\item No preparation of the site of operation is needed prior to deploying the system.
\item No specialized hardware is needed for environment modification or detection of those modifications.
\item Energy consumption for communication is reduced due to the low range requirements.
\item The system is agnostic to the communication hardware, allowing adaptibility to the environment where the system is deployed. For example, RF communication could be used for aerial swarms and sonar for underwater operations.
\item No single point of failure exists as failure of any individual participant does not effect the others.
\end{enumerate}
The disadvantages are the hardware requirements listed in Subsection \ref{subsec:hwreq}.
The algorithm executed by each robot independently consists of repeatedly peforming the following steps:
	\begin{enumerate}
		\item For each pheromone on the robot's virtual pheromone map, decrease its strength $S$, according to \parencite{robinson_decay_2008}, and destroy it if it has completely decayed, i.e. $S\leq0.1$. 
			$$S=0.98 S_{initial} e^{-\lambda t}$$
		\item Every $N_p$-th timestep drop a new pheromone with strength $S=S_{initial}$ at the robot's current position $\vec{r}$.
		\item Every $N$-th timestep, decide on a direction by picking $n$ random directions and for the $k$-th($0\le k<n$) direction calculating a desirability rating $d_k$:
		\begin{enumerate}
			\item Randomly generate $\vec{e_k}$ as a possible future direction with $\theta < 90^{\circ}$, where $\theta$ is the angular distance between $\vec{e_k}$ and the current direction $\vec{e_{c}}$.
			\item With a virtual pheromone map of $m$ pheromones: 
				$$d_k = 1 + \sum_{i=1}^m {\frac{S_i \theta_k}{|\vec{r}-\vec{r_{i}}|^2}}$$ 
				$S_i$ is the strength of the $i$-th pheromone, $\vec{r_{i}}$ is its position, and $\theta_k$ is the angular distance between the pheromone $\vec{p_i}=\vec{r_i} - \vec{r}$ and the considered future direction $\vec{e_k}$.
		\end{enumerate}
	\item Pick the $\vec{e_k}$ with the highest $d_k$ as the new $\vec{e_c}$
	\item Move along the direction of $\vec{e_c}$: $\vec{r} = \vec{r} + v\vec{e_c}$
	\end{enumerate}
\par The algorithm has a complexity of $O(n^2)$ with respect to the number of considered future directions and the number of pheromones on the virtual map of the robot. 
\par No separate fault tolerance mechanisms are implemented.

